\documentclass[11pt,a4paper]{article}

\usepackage[headheight=100pt]{geometry}
\usepackage{fancyhdr}   % custom headers
\usepackage{lipsum}     % insert dummy 'Lorem ipsum' text
\usepackage{amsmath}    % measurement fractions
\usepackage{multicol}   % multi column for ingredients
\usepackage[colorlinks = true, linkcolor = blue, urlcolor  = blue, citecolor = blue, anchorcolor = blue]{hyperref}   % links
\usepackage{siunitx}    % degrees Celsius and measures
\usepackage{times}      % alternate font
\usepackage{textcomp}   % temperature

\pagestyle{fancy}
\fancyhf{}  % clear header and footer
\fancyhead[L]{\fontsize{14}{10} \selectfont Pan Fried Fish with Beans and Cherry Tomatoes}

\begin{document}

This is modified from a CSIRO Well-Being recipe. Rather than aspargus, I suggest brussel sprouts. 
I've also changed the cooking order so that the beans and sprouts are nicely cooked through before
returning the fish. This recipe works best with Gay's rice recipe.

\subsection*{Ingredients}

\begin{itemize}
  \item $ \qty{300}{\gram} $ white fish, skinless, boneless
  \item black pepper
  \item olive oil
  \item $ 3 $ garlic cloves, chopped
  \item $ 2 $ red or green chillies, chopped
  \item $ 1 $ leek, thinly sliced
  \item $ \qty{400}{\gram} $ cherry tomatoes, halved
  \item $ 1 $ tablespoon tomato paste
  \item $ \qty{125}{\gram} $ green beans, halved
  \item $ 8 $ brussel sprouts, halved
  \item $ \qty{40}{\gram} $ baby spinach
  \item $ 1 $ tablespoon capers
  \item $ 1 $ lemon or lime
\end{itemize}

\medskip

\subsection*{Method}

\begin{enumerate}
  \item Cut the fish into $ \qty{4}{\cm} $ chunks. Season well. Heat a deep non-stick frying pan over medium to high heat. Add 1 teaspoon of oil. Add fish and cook for 2 minutes each side or until the fish just starts to colour. Remove to a plate.
  \item Add 1 teaspoon oil to the pan with garlic, leek and chilli, sauté 2 minutes until the leek is soft. Add the beans and brussel sprouts until nearly cooked. Add tomatoes, paste and 1 cup water, bring to a gentle simmer. Cover and simmer for 2-3 minutes. Return the fish, poking into the sauce. Cook until the fish is cooked throughout. 
  \item Remove from the heat, stir in spinach and sprinkle with capers. Squeeze over lemon juice and serve over rice.
\end{enumerate}

\href{https://www.totalwellbeingdiet.com/au/recipes/quick-easy-recipes/one-pan-fish-stew-with-beans-cherry-tomatoes}{One-pan fish stew with beans \& cherry tomatoes}

\end{document}
