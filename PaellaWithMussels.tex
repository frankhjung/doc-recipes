\documentclass[11pt,a4paper]{article}

\usepackage{fancyhdr} % custom headers
\usepackage[headheight=100pt]{geometry}
\usepackage{lipsum}   % insert dummy 'Lorem ipsum' text
\usepackage{amsmath}  % measurement fractions
\usepackage{multicol} % multicolumn for ingredients
\usepackage{hyperref} % links
\usepackage{times}    % alternate font

\pagestyle{fancy}
\fancyhf{}  % clear header and footer
\fancyhead[L]{\fontsize{14}{10} \selectfont Paella with Mussels}

\begin{document}

\subsection*{Ingredients}

\begin{multicols}{2}

\begin{itemize}
  \item $ 60 $ ml ($\frac{1}{4} $ cup) olive oil
  \item $ 1 $ small red onion, peeled and roughly chopped
  \item $ 1 $ red capsicum, seeded and roughly chopped
  \item $ 5 $ garlic cloves, peeled
  \item $ 1 $ teaspoon smoked Spanish paprika
  \item $ 1 $ teaspoon sweet paprika
  \item $ 2 $ medium tomatoes, chopped
  \item $ 1 $ generous pinch saffron
\end{itemize}

\columnbreak

\begin{itemize}
  \item $ 2 $ tablespoons olive oil, extra
  \item $ 250 $ grams short-grain or risotto rice 
  \item $ 2 $ bay leaves
  \item $ 3 - 4 $ sprigs fresh thyme
  \item $ 700 $ ml water
  \item salt and pepper, to taste
  \item $ 500 $ grams mussels, scrubbed and debearded
  \item lemon wedges, to serve
\end{itemize}

\end{multicols}

\medskip

\subsection*{Method}

\begin{enumerate}
  \item Heat the olive oil in a wide-based fry pan or paella pan over medium heat. When hot, add the onion and cook well until it starts to brown, then add the capsicum and garlic and continue to cook over medium heat until softened. Tip in both the paprika's, stir and cook lightly for a few seconds (don't overdo it or they will burn). Add the tomato and saffron, and cook a couple of minutes longer. Remove from the pan and puree to a paste.
  \item In the same pan, heat another $1 - 2$ tablespoons olive oil and fry the rice well until starting to colour a bit. Tip in the onion paste, the bay leaves and thyme and fry for about $5$ minutes.
  \item Stir in the water and pepper, and possibly some salt, taking care not to over salt as the mussels will release its salty brine into the rice later. Bring to a simmer, then turn the heat right down. Cook the rice for about $25$ minutes, without stirring, adding a splash of water if it's drying out. It’s a good idea to let a crust form on the bottom, though the line between a nice crust and a burnt one is fine.
  \item When the rice is just about cooked and the moisture has mostly been absorbed (it shouldn't be soggy), add the mussels, pushing them well down into the rice. Continue to cook until the mussels open; you can cover with a lid to help steam them open, if you like. Once all the mussels have opened, remove from the heat and let the rice sit in a warm place for $5 - 10$ minutes before serving with lemon.
\end{enumerate}

\end{document}
