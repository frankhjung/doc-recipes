\documentclass[11pt,a4paper]{article}

\usepackage[headheight=100pt]{geometry}
\usepackage{fancyhdr}   % custom headers
\usepackage{amsmath}    % measurement fractions
\usepackage{multicol}   % multi column for ingredients
\usepackage[colorlinks = true, linkcolor = blue, urlcolor  = blue, citecolor = blue, anchorcolor = blue]{hyperref}   % links
\usepackage{siunitx}    % degrees Celsius and measures
\usepackage{times}      % alternate font
\usepackage{textcomp}   % temperature

\pagestyle{fancy}
\fancyhf{}  % clear header and footer
\fancyhead[L]{\fontsize{14}{10} \selectfont Sauteed Green Beans with Garlic and Almonds}

\begin{document}

\subsection*{Ingredients}

\begin{itemize}
  \item $ 500 $ \si{\gram} green beans, ends trimmed
  \item $ 1 $ tablespoon seasame oil
  \item $ 1 $ tablespoon peanut or olive oil
  \item $ 1 $ teaspoon salt
  \item $ 1 $ red chilli, quartered lengthways
  \item $ 2 $ teaspoons garlic, chopped
  \item $ \frac{1}{3} $ cup crushed almonds
\end{itemize}

\medskip

\subsection*{Method}

\begin{enumerate}
  \item Make sure the beans are patted dry, then add to a large frying pan along with oil.
  \item Saute beans over medium heat 1-2 minutes.  
  \item Add garlic, salt, chilli and almonds.
  \item Saute 10 minutes or until green beans are easily pierced with a fork. 
  \item Allow to cool slightly before serving.
  \item Drizzle some balsamic glaze (optional).
  \item Sprinkle with sesame seeds (optional).
\end{enumerate}

\href{https://www.lecremedelacrumb.com/sauteed-green-beans-with-garlic-and-almonds/}{www.lecremedelacrumb.com/sauteed-green-beans-with-garlic-and-almonds}

\end{document}
