\documentclass[11pt,a4paper]{article}

\usepackage[headheight=100pt]{geometry}
\usepackage{fancyhdr}   % custom headers
\usepackage{lipsum}     % insert dummy 'Lorem ipsum' text
\usepackage{amsmath}    % measurement fractions
\usepackage{multicol}   % multi column for ingredients
\usepackage[colorlinks = true, linkcolor = blue, urlcolor  = blue, citecolor = blue, anchorcolor = blue]{hyperref}   % links
\usepackage{siunitx}    % degrees Celsius and measures
\usepackage{times}      % alternate font
\usepackage{textcomp}   % temperature

\pagestyle{fancy}
\fancyhf{}  % clear header and footer
\fancyhead[L]{\fontsize{14}{10} \selectfont Name of Recipe}

\begin{document}

\subsection*{Ingredients}

\begin{multicols}{2}

\begin{itemize}
  \item $ \frac{1}{2} $ teaspoon
  \item $ \frac{1}{3} $ tablespoon
  \item $ 1 $ \si{\gram}
  \item salt \& black pepper

\end{itemize}

\columnbreak

\begin{itemize}
  \item $ 1 $ cup
\end{itemize}

\end{multicols}

\medskip

\subsection*{Method}

\begin{enumerate}
  \item heat oven to 150\si{\celsius}
  \item \lipsum[2]
  \item \lipsum[3]
\end{enumerate}

\href{https://en.wikipedia.org/wiki/Recipe}{Source of Recipe}


\end{document}
