% vim: set filetype=plaintex shiftwidth=2 tabstop=2 expandtab:
\documentclass[11pt,a4paper]{article}

\usepackage[headheight=100pt]{geometry}
\usepackage{fancyhdr}   % custom headers
\usepackage{lipsum}     % insert dummy 'Lorem ipsum' text
\usepackage{amsmath}    % measurement fractions
\usepackage{multicol}   % multi column for ingredients
\usepackage[colorlinks = true, linkcolor = blue, urlcolor  = blue, citecolor = blue, anchorcolor = blue]{hyperref}   % links
\usepackage{siunitx}    % degrees Celsius and measures, see https://texdoc.org/serve/siunitx/0
\usepackage{times}      % alternate font
\usepackage{textcomp}   % temperature

\pagestyle{fancy}
\fancyhf{}  % clear header and footer
\fancyhead[L]{\fontsize{14}{10} \selectfont Loquat Chutney}

\begin{document}

\subsection*{Ingredients}

\begin{multicols}{2}

\begin{itemize}
  \item $ 24 $ loquats, skins on, pitted, chopped
  \item $ \frac{1}{2} $ onion, finely chopped
  \item $ 3 $ garlic cloves, peeled, finely chopped
  \item $ 1 $ tablespoon ginger, grated
  \item $ \frac{1}{2} $ teaspoon salt
  \item $ 3 $ tablespoons apple cider vinegar
  \item $ 1 $ tablespoon honey, maple  or sugar syrup
\end{itemize}

\columnbreak{}

\begin{itemize}
  \item $ \frac{1}{2} $ teaspoon chilli flakes
  \item $ \frac{1}{2} $ teaspoon coriander seeds
  \item $ \frac{1}{2} $ teaspoon mustard seeds
  \item $ \frac{1}{2} $ teaspoon cumin powder
  \item $ \frac{1}{2} $ teaspoon turmeric powder (with pinch of black pepper)
  \item $ 1 $ teaspoon curry powder
\end{itemize}

\end{multicols}

\medskip

\subsection*{Method}

\begin{enumerate}
  \item \textbf{Prepare Ingredients} Wash, pit, and chop loquats. Finely chop
  onion, grate ginger, and crush garlic.

  \item \textbf{Toast Spices} Heat a splash of oil in a small pan over low heat.
  Add coriander and mustard seeds; let them pop and release aroma (1--2
  minutes).

  \item \textbf{Cook Aromatics} Add onion, garlic, and ginger. Stir-fry for 2--3
  minutes until onions soften.

  \item \textbf{Add Spices} Stir in chilli flakes, cumin, turmeric (with black
  pepper), curry powder, and cardamom. Mix well to coat.

  \item \textbf{Simmer Chutney} Add chopped loquats, salt, syrup, and apple
  cider vinegar. Stir and simmer gently for 6--7 minutes, stirring occasionally.

  \item \textbf{Cool \& Store} Let chutney cool completely. Transfer to a
  jar and refrigerate. Keeps for about 1 week.

\end{enumerate}

\href{https://dishingoutplants.com/10-minute-nispero-chutney-spicy-loquat-recipe/}{10-Minute Níspero Chutney (Spicy Loquat Recipe)}

\end{document}
