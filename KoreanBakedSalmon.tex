\documentclass[11pt,a4paper]{article}

\usepackage[headheight=100pt]{geometry}
\usepackage{fancyhdr}   % custom headers
\usepackage{amsmath}    % measurement fractions
\usepackage{multicol}   % multi column for ingredients
\usepackage[colorlinks = true, linkcolor = blue, urlcolor  = blue, citecolor = blue, anchorcolor = blue]{hyperref}   % links
\usepackage{siunitx}    % degrees Celsius and measures
\usepackage{times}      % alternate font
\usepackage{textcomp}   % temperature

\pagestyle{fancy}
\fancyhf{}  % clear header and footer
\fancyhead[L]{\fontsize{14}{10} \selectfont Korean Baked Salmon}

\begin{document}

\subsection*{Ingredients}

\begin{multicols}{2}

\paragraph*{For the marinade}

\begin{itemize}
  \item $ 1 $ tablespoon lime juice
  \item $ 1 $ teaspoon gochugaru
  \item $ 3 $ tablespoon gochujang
  \item $ 2 $ tablespoon light soy sauce
  \item $ 1 $ tablespoon sesame oil
\end{itemize}
  
\columnbreak

\begin{itemize}
  \item $ 4 $ salmon fillets
  \item $ 3 $ spring onions, thinly sliced
  \item $ 1 $ tablespoon toasted sesame seeds
\end{itemize}

\end{multicols}

\medskip

\subsection*{Method}

\begin{enumerate}
  \item heat oven to 200\si{\celsius}
  \item Whisk all of the marinade ingredients in a bowl until smooth. Cover the salmon fillets with the marinade. You can allow to marinate for up to 30 minutes or start cooking immediately.
  \item Once your oven is up to heat, turn on the grill to its highest setting. Place the salmon fillets on a lightly greased rack in the oven, flesh side up and grill for 6 to 10 minutes or until the salmon is nicely browned and opaque in the centre.
  \item Serve sprinkled with the toasted sesame seeds and garnish with the spring onions. Lime wedges can be squeezed over it to taste.
\end{enumerate}

\href{https://greatcurryrecipes.net/2022/08/31/baked-salmon-korean-style/}{Baked Salmon Korean Style by AuthorDan Toombs}

\end{document}
