% vim: set filetype=plaintex shiftwidth=2 tabstop=2 expandtab:
\documentclass[11pt,a4paper]{article}

\usepackage[headheight=100pt]{geometry}
\usepackage{fancyhdr}   % custom headers
\usepackage{lipsum}     % insert dummy 'Lorem ipsum' text
\usepackage{amsmath}    % measurement fractions
\usepackage{multicol}   % multi column for ingredients
\usepackage[colorlinks = true, linkcolor = blue, urlcolor  = blue, citecolor = blue, anchorcolor = blue]{hyperref}   % links
\usepackage{siunitx}    % degrees Celsius and measures, see https://texdoc.org/serve/siunitx/0
\usepackage{times}      % alternate font
\usepackage{textcomp}   % temperature

\pagestyle{fancy}
\fancyhf{}  % clear header and footer
\fancyhead[L]{\fontsize{14}{10} \selectfont Spicy Fried Beans}

\begin{document}

\subsection*{Ingredients}

\begin{itemize}
  \item $ \qty{500}{\gram} $ green beans
  \item $ 1 $ tablespoon oil
  \item $ 1 $ medium onion, finely chopped
  \item $ 1 $ tablespoon grated ginger
  \item $ 1 $ teaspoon ground turmeric
  \item $ 1 $ teaspoon garam masala
  \item $ \frac{1}{2} $ teaspoon chilli powder
  \item salt
  \item 12 cherry tomatoes, chopped
  \item lemon juice
\end{itemize}

\medskip

\subsection*{Method}

\begin{enumerate}
  \item Top and tail beans, remove strings if necessary and cut them into bite size pieces. 
  \item Heat the ghee in a saucepan and fry the onions and ginger over a medium low heat until the onions are golden.
  \item Add the turmeric, garam masala, chilli powder and salt and fry for 2 minutes.
  \item Add tomatoes or water and cook, stirring, until the tomatoes are cooked to a pulp and most of the liquid evaporates.
  \item Add beans and stir well. Partially cover the pan with a lid and cook until the beans are just tender.
  \item Stir in lemon juice to taste.
\end{enumerate}

\href{https://practicallydaily.blogspot.com/2009/08/spicy-fried-beans.html}{Spicy fried beans}

\end{document}
