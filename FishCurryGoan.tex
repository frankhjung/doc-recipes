\documentclass[11pt,a4paper]{article}

\usepackage[headheight=100pt]{geometry}
\usepackage{fancyhdr}   % custom headers
\usepackage{lipsum}     % insert dummy 'Lorem ipsum' text
\usepackage{amsmath}    % measurement fractions
\usepackage{multicol}   % multi column for ingredients
\usepackage[colorlinks = true, linkcolor = blue, urlcolor  = blue, citecolor = blue, anchorcolor = blue]{hyperref}   % links
\usepackage{siunitx}    % degrees Celsius and measures
\usepackage{times}      % alternate font
\usepackage{textcomp}   % temperature

\pagestyle{fancy}
\fancyhf{}  % clear header and footer
\fancyhead[L]{\fontsize{14}{10} \selectfont Goan Fish Curry}

\begin{document}

\subsection*{Ingredients}

\begin{multicols}{2}

Curry Paste

\begin{itemize}
  \item kashmiri chilli
  \item fenugreek
  \item tumeric
  \item coriander, cumin \& ground cloves
  \item fresh garlic \& ginger
  \item tamarind
\end{itemize}

\columnbreak

Curry Sauce

\begin{itemize}
  \item black mustard seeds
  \item chilli poweder
  \item coconut milk
  \item tomato pulp
  \item whole tomato
  \item tomato paste
  \item sugar
\end{itemize}

\end{multicols}

\medskip

\subsection*{Method}

\begin{enumerate}
  \item make the curry paste first, add a bit of water, blend all ingredients
  \item blitz onion until it is a purée
  \item add tomato paste \& purée, cook off the curry paste to get rid of extra water
  \item add coconut milk
  \item simmer for 5 minutes
  \item add fish, cook until done 3~4 minutes
  \item serve on rice garnish with fresh coriander \& chilli
\end{enumerate}

\href{https://www.recipetineats.com/goan-fish-curry-indian/}{www.recipetineats.com/goan-fish-curry-indian}


\end{document}
